\section{Usage}
\label{g14:sec:usage}

\subsection{System requirements}
\label{g14:sec:usage:requirements}
At the time of writing, the server runs with \textit{Python} 3.6, 3.7 and 3.8. The file
\texttt{requirements.txt} lists all required modules.
There is also a \texttt{README.md} for the most recent instructions.

\subsection{Installation}
\label{g14:sec:usage:instllation}
First you have to create a file \texttt{src/acomp/config.py} that conatins
a random \texttt{SECRET\_KEY}, a \textit{SQLALCHEMY\_DATABASE\_URI}
(e.g. \texttt{SQLALCHEMY\_DATABASE\_URI = 'sqlite:////tmp/acomp.sqlite3'}) and
a \texttt{SESSION\_TYPE} (e.g. \texttt{SESSION\_TYPE = 'filesystem'}).
\textit{Note:} Make sure that you initiise the \texttt{SECRET\_KEY} with a secure random function.
It must not be changed afterwards, otherwise existing data get corrupted.
You may want to set \texttt{NLTK\_DATA} to use a custom location, e.g. to the data
managed by your distobution package manager, for the natural language toolkit files.

Run the following commands in the \texttt{src} directory to install all dependencies and
database initialisation:

\begin{verbatim}
pip install -r requirements.txt
flask nltk-data
flask db initrequirements
flask db migrate
flask db uprade
flask run
\end{verbatim}

Use \texttt{uwsgi -w app:app --socket 0.0.0.0:5000} and a webserver for production environments.

You can also use your package manager or \texttt{virtualenv} to install them.

If you have any existing images to import you can run
\texttt{flask prefill /my/image/folder}. By default only \texttt{.jpg} and \texttt{.png}
are allowed. This can be changed in the \texttt{ACOMP\_ALLOWED\_FILE\_EXT} setting. The
images are stored with their hash as filename in the \texttt{ACOMP\_OBJ\_DIR} location.
If you'd like to provide tags in another language than English, you may change
\texttt{ACOMP\_LANGUAGE\_DEFAULT} in \texttt{src/acomp/config.py} to the
ISO-639-1 \todo{reference} code of your preferred language.
However, all messages are not yet provided in other languages than English,
and the response to your provided tags may be confusing.
We recommend playing in English, which is also the default.

\subsection{Docker}
\label{g14:sec:usage:docker}
Docker provides an easy way to deploy platform independent containers on Linux.
Run the following commands in the root directory of the source code:

\begin{verbatim}
docker build -t acomp src/
docker run --rm -v \$PWD/src:/app -p127.0.0.1:5000:5000 acomp
\end{verbatim}

You can use environment variables to pass configuration options.

You can now access the service in your favorite browser at
\texttt{http://localhost:5000/}.

Note: With this example any changed data will be lost when you stop the
container! There is a \texttt{docker-compose.yml} as inspiration for production deployments.

\subsection{Exhaustive configuration reference}
\begin{verbatim}
# Flask config to run in debug mode
DEBUG = False
# Flask config to handle passwords longer than 72 bytes
BCRYPT_HANDLE_LONG_PASSWORDS = True
# Flask config to print SQL statements
SQLALCHEMY_ECHO = False
# Flask config to disable tracking of objects by
SQLALCHEMY_TRACK_MODIFICATIONS = False
# allowed file extensions for the prefill command
ACOMP_ALLOWED_FILE_EXT = ['.jpg', '.png']
# output directory of the prefill command
ACOMP_OBJ_DIR = "static/images"
# session timeout during the game
ACOMP_LIFETIME_USER = 70
# default language for user tags
ACOMP_LANGUAGE_DEFAULT = 'en'
# required balance of corrent answers to pass the entry quiz
ACOMP_QUIZ_POINTS = 2
# time limit in classic mode (in seconds)
ACOMP_CLASSIC_TIMELIMIT = 30
# maximum ratio
ACOMP_CLASSIC_RATIO = 0.2
# amount of tags that are shown as already mentioned
ACOMP_CLASSIC_FORBIDDEN_TAGS = 5
# time limit in captcha mode (in seconds)
ACOMP_CAPTCHA_TIMELIMIT = 30
# number of images shhown in captcha mode
ACOMP_CAPTCHA_NUM_IMAGES = 4
# number of tags shown in captcha mode
ACOMP_CAPTCHA_NUM_TAGS = 3
# number of users shown in the highscore
ACOMP_NUM_HIGHSCORE = 5
# amount of user that are needed ro reduce the score for already mentioned tags
ACOMP_NUM_LEV1 = 1
# amount of user that are needed to mark a tag as already mentioned by others
ACOMP_NUM_LEV2 = 2
# default directory for the natural language toolkit data
NLTK_DATA = "/usr/share/nltk_data"
\end{verbatim}
