\section{Usage}
\label{gXLII:sec:usage}  % TODO Adopt group prefix

\subsection{System requirements}
\label{gXLII:sec:usage:requirements}  % TODO Adopt group prefix
At the time of writing, the server runs with \textit{Python} 3.6, 3.7 and 3.8. The file
\texttt{requirements.txt} lists all required modules.
There is also a \texttt{README.md} for the most recent instructions.

\subsection{Installation}
\label{gXLII:sec:usage:instllation}  % TODO Adopt group prefix
First you have to create a file \texttt{src/acomp/config.py} that conatins
a random \texttt{SECRET\_KEY}, a \textit{SQLALCHEMY\_DATABASE\_URI}
(e.g. \texttt{SQLALCHEMY\_DATABASE\_URI = 'sqlite:////tmp/acomp.sqlite3'}) and
a \texttt{SESSION\_TYPE} (e.g. \texttt{SESSION\_TYPE = 'filesystem'}).
You may want to set \texttt{NLTK\_DATA} to use a custom location, e.g. to the data
managed by your distobution package manager, for the natural language toolkit files.

Run the following commands in the \texttt{src} directory to install all dependencies and
database initialisation:

\begin{verbatim}
pip install -r requirements.txt
flask nltk-data
flask db initrequirements
flask db migrate
flask db uprade
flask run
\end{verbatim}

Use \texttt{uwsgi -w app:app --socket 0.0.0.0:5000} and a webserver for production environments.

You can also use your package manager or \texttt{virtualenv} to install them.

If you have any existing images to import you can run
\texttt{flask prefill /my/image/folder}.
If you'd like to provide tags in another language than English, you may change
\texttt{ACOMP\_LANGUAGE\_DEFAULT} in \texttt{src/acomp/config.py} to the
ISO-639-1 \todo{reference} code of your preferred language.
However, all messages are not yet provided in other languages than English,
and the response to your provided tags may be confusing.
We recommend playing in English.

\subsection{Docker}
\label{gXLII:sec:usage:docker}  % TODO Adopt group prefix
Docker provides an easy way to deploy platform independent containers on Linux.
Run the following commands in the root directory of the source code:

\begin{verbatim}
docker build -t acomp src/
docker run --rm -v \$PWD/src:/app -p127.0.0.1:5000:5000 acomp
\end{verbatim}

You can use environment variables to pass configuration options.

You can now access the service in your favorite browser at
\texttt{http://localhost:5000/}.

Note: With this example any changed data will be lost when you stop the
conatiner! There is a \texttt{docker-compose.yml} as inspiration for production deployments.
